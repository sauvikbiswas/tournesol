\chapter{Functions}
In calculus, one of the first thing a student has to understand is the concept of functions. Yet, it is one of the most overlooked part of high-school mathematics.

\begin{definition}[Function]
A function is a relation between a set of inputs to a set of outputs. For a given input, it has one and only one particular output
\end{definition}

It is often written in the following form -- $y = f(x)$, where $x$ is the input and $y$ is the output.

The letters $x$, $y$, and $f$ are arbitrary placeholders. You can substitute these with anything. For many real-world applications, they are meant to be substituted in order to obtain tangible values. In fact, they can be substituted with a different set of placeholders without losing the intended meaning.

Let us look at a function that doubles the input.

\begin{equation}
y = f(x) = 2x
\end{equation}

Here, $x$ can take any number. It could be an integer, say, $x = 2$. Then,

\begin{equation}
y = f(x) = 2x = 2 \times 2 = 4
\end{equation}

What if $x$ is a complex number, say, $x = 5 + 3i$? In that case,

\begin{equation}
y = f(x) = 2x = 2 \times (5 + 3i) = 10 + 6i
\end{equation}

It must be noted that, no matter what we plug-in as $x$, the output (or $y$) is the same for a given value of $x$.

Let us consider a very different kind of function. What if we want to compute the number of unique letters a word has? Let us call this function $c$.

\begin{equation}
y = c(x) = \text{number of unique letters in $x$}
\end{equation}

Here are some computed cases --

\begin{equation}
x = \text{``HELLO"}; y = c(x) = c(\text{``HELLO"}) = 4
\end{equation}
\begin{equation}
x = \text{``OVER"}; y = c(x) = c(\text{``OVER"}) = 4
\end{equation}
\begin{equation}
x = \text{``WATERMELON"}; y = c(x) = c(\text{``WATERMELON"}) = 9
\end{equation}

Just like the earlier example, we observe that the output for a given input is fixed. $c(\text{``Hello"})$ will never be 2, 3, 5, 6, or anything other than 4.

There is another aspect that must be highlighted. The input to this function cannot be anything under the sun. For example, $c(x)$ makes no sense when $x$ is a real number, a complex numer, or even a word from the Indian languages. This restriction to the type of input a function can accept gives rise to the domain of the function.

\begin{definition}[Domain of a function]
The domain of a function is the set of input values for which the function is defined.
\end{definition}

The output of $c(x)$ will invariably be a positive integer. You can't have it return a floating point number. I can't think of any word whose unique letter count is 3.14156. Can you?
